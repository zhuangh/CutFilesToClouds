%\subsection{Fault Tolerance}
Two big issues here need be more focused. One is that we cannot lose anything about the mapping strucuture between file and related segments. We solve this by using a replicated key value distributed storage server. As one trusted bridge server just talks to one key value storage server one moment in time, the replicated storage server will help syncrhonize writing key value pair by communicating with each other in order to prevent some storage server crashing. This method will enormously get the risk out. 
The second one is to keep data consistent during the period of generating the segment. For example, server may fail in trasmitting segments to the cloud. To fix this, a thrad can be used to monitor and periodically check whether the segment has been put on the external cloud storage. If it has been there, the server begin to update the item to array in the mapping structure for this segment. If it is not, the trustd bridge server shall resend current current segment. When learning that all the segments for one file have been pushed to the cloud, the json object describing one file with corresponding segments will be put into the key value storage server. The user will not get the information that the file has been uploaded successfully till we get the commit acknowledgement from the key value storage server.
