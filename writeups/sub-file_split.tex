

File Partitioning is the key point of this system and it inclues two main 
problems: first one is how to partition the file and the 
other one is how fine-grained the file need to be partitioned.
For the first one, 
when a file is uploaded to our server, we need to dispatch the file to different
cloud storages. Our strategy is to read the 
file and truncate content on the fly, so that
the large files do not occupy the large chunk of memory.
For implementation, we also use JSON in Boost to  
control hierachical key value format, e.g. the order of segments for a file,
the mapping from file to segments and different cloud services.
First, the process get the parameter vector to specify how many segments for this
file, and what the sizes are respectively.
During the processs of spliting, it also invokes \emph{GetSegmentName()} to get a 
encrypted name for such segement, which appear in the cloud storage. Others cannot 
judge the file by name. The method of \emph{GetSegmentName()} is demonstrated in Sec. \ref{naming_hash}.
The pseudo code is shown in Fig. \ref{fig:fsplit}.

%\setlength{\belowcaptionskip}{0pt minus 10pt}
\begin{figure*}[hbt]
\centering
\begin{lstlisting}
fp = fopen(file, "rb"):
for (int it = 0 ; it < chunk_size_array.size ; it++){
  buffer = malloc(chunk_size_array[it]);
  file_name = GetSegmentName(user_key)
  // treating every file as binary 
  segment_file = fopen(file_name, "wb")
  fwrite(buffer,1,current , segment_file);
  fclose(segment_file);
  free(buffer);
  // use hierachical Key Value pair by JSON 
  // It provide a file contains what segments,
  // their order, and position.
  // Simplify the process by register_to_key_value() as follows
  register_to_key_value(user, cloudID, file , segment_order , segment_name);
}
\end{lstlisting}
\caption{Pesudo Code of File Spliting}
\label{fig:fsplit}
\end{figure*}

For the merge of segments from different clouds, we fetch them from their repositories at first, 
Following the records from our key value storage, assemble them together accordingly.

\begin{comment}
\begin{figure}
\centering
\begin{lstlisting}
// Download the segments in order througth Key Value storage's guide
std::list<string> l;
int index = 1;
BOOST_FOREACH(ptree::value_type& v, pt.get_child("array")){
   counter = boost::lexical_cast<string>(index++);
   segmentName = v.second.get_child(counter).data();
   key = FormKey(user_key, segmentName);
   GetResponse response = Get(key);
   // find out which cload ID
   extid = response.value;
   Download(username, segmentName, cloudID);
   l.push_back(segmentName);    
}

file_name
fp = fopen(file, "rb"):
for (int it = 0 ; it < chunk_size_array.size ; i++){
  buffer = malloc(chunk_size_array[it]);
  file_name = GetSegmentName(user_key)
  // treating every file as binary 
  segment_file = fopen(file_name, "wb")
  fwrite(buffer,1,current , segment_file);
  fclose(segment_file);
  free(buffer);
  // use hierachical Key Value pair by JSON 
  register_to_key_value(user, file , segment_order , segment_name);
}
\end{lstlisting}
\caption{Pesudo Code of File Merge}
\end{figure}
\end{comment}



\begin{figure*}
\centering
\begin{lstlisting}
    key                                 Value
    ---------------------------------------------------
    ${username}_clouds             (JSON) clouds
    ${username}_files              list<filename>
    ${username}_${filename}        list<segment_name>
    ${username}_${segment_name}    cloud_id
    ${username}_key                any string
    ---------------------------------------------------
\end{lstlisting}
\caption{Key-Value Storage Format}
\label{fig:kv_format}
\end{figure*}


\begin{figure*}
\centering
\begin{lstlisting}
	    early_cloud  {
	    "clouds":[
	    {"id":0, "type":"dropbox", "token1":"12345",  "token2":"67890" },
	    {"id":1, "type":"dropbox", "token1":"12345",  "token2":"67890"},
	    {"id":2, "type":"google" , "token1":"12345",  "token2":"67890"}
	    ]
	    }
	    early_files             list<"file1.avi","file2.txt","file3.jpg"...>
	    early_file1.avi         list<"seg1.seg","seg2.seg","seg3.seg"...>
	    early_file2.txt         list<"seg1.seg","seg2.seg","seg3.seg"...>
	    early_seg1.seg          0
	    early_seg2.seg          1
	    early_seg3.seg          2
\end{lstlisting}
\caption{Example of Our Key Value Pair}
\label{fig:example-kv}
\end{figure*}
																																																																																																																																									      




