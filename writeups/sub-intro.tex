
\IEEEPARstart{N}{owadys}, there are plenty of cloud storage service providers on the Internet, such as Dropbox, Google Drive, Skydrive or Amazon cloud storage. Most of them have easy-to-use user interface and support most of desktop and mobile platforms. However, under the mask of accessibility and reliability, we never know whether these service providers will analyze your contents or not even though they promised not to do so in the term of service. Take Google Drive for example, it indexes every file uploaded to its servers to optimize the searching efficiency. Even for those image files they could do OCR to filter out the important text content inside, and that is how Evernote works for the image in notes.

One of the most popular method in data mining is using term frequency–inverse document frequency product (TF-IDF)\cite{TF-IDF}, which reflects the importance of a word in a document. By computing TF-IDFs for each file, the system could easily do data anaylsis for all the files that belong to specific users and learn what they like, what they working at, or anything personal. 

We need a service that could take advantage from the convenience of cloud storage without loosing our privacy. Our approach is striping each single file into several fragments using a hashing algorithm with user-specified key and store those striped files in different cloud storage. Scattering the file could dramatically destroy the reliability of TF-IDFs since the weight in each fragment file could not efficiently reflect the whole content. What the cloud servers could see are encrypted file names and re-ordered file fragments.
